
\documentclass[11pt]{beamer}
\mode<presentation>
\usepackage{tikz}
\usepackage{siunitx}
\usepackage{subfig}
\usetikzlibrary{arrows}
\tikzstyle{block}=[draw opacity=0.7,line width=1.4cm]
\usepackage{gensymb}
\usepackage{Sweave}
\usepackage{pdflscape}
\usepackage{tabularx}% <-- added
\usepackage{textcomp}
%\usefonttheme{serif} 
\usefonttheme{professionalfonts} 
\usetheme{Madrid}
\usecolortheme{beaver}
\setbeamertemplate{navigation symbols}{}
\defbeamertemplate*{footline}{shadow theme}{%
 \ifnum \insertpagenumber=1
      \leavevmode%
      \hbox{%
      \begin{beamercolorbox}[wd=\paperwidth,ht=2.25ex,dp=1ex,center]{}%
       
      \end{beamercolorbox}}%
      \vskip0pt%
    \else
\leavevmode%
\hbox{\begin{beamercolorbox}[wd=.5\paperwidth,ht=2.5ex,dp=1.125ex,leftskip=.3cm plus1fil,rightskip=.3cm]{author in head/foot}%
    \usebeamerfont{author in head/foot}\hfill\insertshortauthor
\end{beamercolorbox}%

\begin{beamercolorbox}[wd=.5\paperwidth,ht=2.5ex,dp=1.125ex,leftskip=.3cm,rightskip=.3cm plus1fil]{title in head/foot}%
    \usebeamerfont{title in head/foot}\insertshorttitle\hfill%
\insertframenumber\,/\,\inserttotalframenumber
\end{beamercolorbox}}%
\vskip0pt%
  \fi
}
\title[VST for image-based compound profiling features]{Variance Stabilizing Transformations for image-based compound profiling features}
\author{LEonard Wafula}
\vspace{2cm}
% \date{August 01, 2017}
\date{\today}
\begin{document}
\begin{frame}
\titlepage
\end{frame}

\begin{frame}{Introduction}
\textbf{\underline{Image-based multi-parameteric compound profiling features}} \\ 
\vspace{2mm}
$\Box$ Biological method used as a proxy for distinguishing compounds in the drug discovery chain using a range of features extracted
from image-based assays by applying High Throughput Microscopy (HTM) \\
$\Box$ The features provide information on \\
\begin{enumerate}[i.]
\item{Intracellular biomarkers: texture, intensity, spatial distribution etc}
\item{Cells: shape, geometry, quantity .. .}
\end{enumerate}
$\Box$ Why
\begin{enumerate}[i.]
\item{understand how compounds induce their desired properties and describe their mechanisms of action}
\item{preferentially identify highly specific compounds having a desired effect on a given biological target}
\item{early detection of undesired compound effects on cells + cellular activity: toxicity}
\end{enumerate}
\vspace{4mm}
\end{frame}

\begin{frame}{.. introduction}
It's all good, but .... Most of these features often \\
\begin{enumerate}[$\Box$]
 \item are highly correlated: need to limit features used for analysis \\
\item have non-normal distributions: mean-variance relationship present
\begin{itemize}
\item multivariate classification methods hugely depend on variance
\item {\includegraphics[scale=0.25]{variability.PNG}}
\end{itemize}
\end{enumerate}
\end{frame}

\begin{frame}{Aim of the analysis}
To assess the:
\begin{enumerate}[I. ]
\item effect of glog transformation on separation of treatment replicates from non-replicates
\item effect of glog transformation on proportion of actively-called treatments
\item performance of a glog transformation on treatments separation when applied at cell- or well-level
\end{enumerate} 
\vspace{4mm}
$\clubsuit$ Treatment: a compound at a given concentration (a compound can have 4 or 5 concentration levels - \SI{1}{\micro M} (microMolar), \SI{3}{\micro M}, \SI{3.34}{\micro M}, \SI{9}{\micro M} and \SI{11.1}{\micro M})
\end{frame}

\begin{frame}{Data, the glog transformation and data pre-processing}
       \begin{columns}
          \column{0.45\textwidth}
             \centering
         \includegraphics[scale=1]{dataview_structured20072017.PNG}
        \column{0.55\textwidth}
         \textcolor{black}{\underline{Data}}
         \begin{enumerate}[i.]
         \item{2 cancer cell lines: Liver \& Colon } 
         \item{a plate had btwn 134909 \& 281177 (152679 \& 330117) in $1^{st}$($2^{nd}$) data}
        \item {a well had btwn 73 \& 1812 (95 \& 2120) cells in the $1^{st}$($2^{nd}$) data}
        \item 311 compounds including DMSO control
        \item {we tested a total of 1253 treatments}
        \item 462 features extracted from each cell
        \end{enumerate}
         \end{columns}
\end{frame}

\begin{frame}{..Data, the glog transformation and data pre-processing}
       \begin{columns}
          \column{0.5\textwidth}
             \textcolor{black}{\underline{Glog transformation }}
\begin{enumerate}[$\Rsh$]
\item formula\\
\begin{equation*}
\begin{split}
z=\mbox{Log}(y-\alpha+\sqrt{(y-\alpha)^2+\lambda})
\end{split}
\end{equation*}
\item where
\begin{enumerate}[$\bullet$]
\item z: glog-transformed data
\item y: untransformed data
\item $\alpha$: feature mean across DMSO controls
\item $\lambda$: transformation parameter
\end{enumerate}
\end{enumerate}
             \textcolor{black}{\underline{Data pre-processing}}
             \begin{enumerate}[$\Rsh$]
\item Aggregation - calculating mean for each feature per well
\end{enumerate}
\column{0.5\textwidth}
\begin{enumerate}[$\Rsh$]
\item Normalization - 
$$\frac{\mbox{feature}_{value}-\mbox{mean.feature}_{DMSO}}{\mbox{pooled.SD.feature}_{across.plates}}$$
\item Feature selection 
\begin{enumerate}[$\bullet$]
\item MRMR: identify set of features with low pairwise correlation \& high reproducibility among replicates.
\item AUC value for btwn 2-75 features
\item optimal feature: maximizes separation of treatment replicates within 1 Std error of AUC
\end{enumerate}
\item Active calling: treatments with $\ge 50$\% active replicates
\end{enumerate}
 \end{columns}
\end{frame}

\begin{frame}{Methodology}
\begin{columns}
\column{0.5\textwidth}
$\star$ \underline{Hotelling's T$^{2}$ method}
\begin{enumerate}[$\leadsto$]
\item measures difference in 2 multivariate means
\item formula \\
\begin{flushleft}
\begin{equation*}
\begin{split}
T^2 =
\frac{(\mathbf{\bar{X}}_1-\mathbf{\bar{X}}_2)\prime(\mathbf{\bar{X}}_1-\mathbf{\bar{X}}_2)}{\boldsymbol{S}_p(\frac{1}{n_1}+\frac{1}{n_2})}
\end{split}
\end{equation*}
\end{flushleft}
\item normality assumptions for optimal results
\item Only actively-called treatments in pre- \& post-transformation used
\item + shift in T$^2$ distribution indicate improved treatment separation
\end{enumerate}
\column{0.5\textwidth}
$\star$ \underline{AUC method} \\
2-steps involved in AUC-calculation
\begin{enumerate}[$\leadsto$]
\item Pearson correlation btwn pairs of replicates (\& non-replicates) were calculated \& distributions plotted
\item separation btwn the 2 distribution quantified by constructing an ROC curve using a series of correlation thresholds \& calculating an AUC value
\item transformations leading to higher AUC values $\dashrightarrow$ improved treatments separation compared to corresponding untransformed data
\end{enumerate}
\end{columns}
\end{frame}
\begin{frame}{Results:  EDA}
\begin{tabular}{ p{.2\textwidth}  p{.15\textwidth}  p{.1\textwidth} }
\multicolumn{3}{l}{No. of replicates(\% of sample treatments)}\\   
9 (\%) & 36 (\%) & 45 (\%) \\ \hline
1192(95.208) & 28(2.236) & 32(2.556) \\
\end{tabular}
\vspace{3mm}
\begin{enumerate}[$\bigstar$]
\item DMSO control replicated across 1512 wells
\item For both data sets
\item Implications \\
\begin{enumerate}[$\star$]
\item For calculation of Hotelling's T$^2$, a limited number of selected features was used to maximize its power
\item 10 highest ranked features from MRMR used to calculate T$^2$
\end{enumerate}
\end{enumerate}
\end{frame}

\begin{frame}{Transformations effects on treatments separation}
\underline{Prologue} \\ \vspace{3mm}
$\diamond$ Only glog transformations of $\lambda$ equal to 0.1 and, 0.5 to 25 at 0.5 interval investigated for both T$^2$ and AUC methods \\ \vspace{3mm}
$\diamond$ Each transformed data compared to its corresponding untransformed data defined by actively-called treatments present both pre- and post-transformation \\ \vspace{3mm}
$\diamond$ Improved treatments separation shown by +ve shifts in distribution (and/or associated statistics) of T$^2$ for transformed compared to untransformed, and/or higher AUC values \\ \vspace{3mm}
$\diamond$ Results presented for the first cell line only since results largely led to similar conclusions for both cell lines
\end{frame}

\begin{frame}{Transformations effects on treatments separation - T$^2$}
$\divideontimes$ Presence of very high [very different] and very low [highly similar] values \\ \vspace{3mm}
\begin{columns}
\column{0.5\textwidth}
$\divideontimes$ Some led to slight but negligible improvements (e.g $\lambda = 10.5$)
\includegraphics[width=1\linewidth]{LOG_matched__hotellings_allactiveTRTS_lambda=10point5.pdf}

\column{0.5\textwidth}
$\divideontimes$ Others led to no improvements (e.g $\lambda = 10$)
\includegraphics[width=1\linewidth]{LOG_matched__hotellings_allactiveTRTS_lambda=10.pdf}
\end{columns}
\end{frame}

\begin{frame}{Transformations effects on treatments separation - AUC}
\begin{columns}
\column{0.65\textwidth}
\includegraphics[width=.95\linewidth]{matched_auc_all_lambdas.pdf}
\column{0.35\textwidth}
\begin{enumerate}[ ] 
\item \hspace{-1cm} $\bullet$ high AUC b4-transformation
\item \hspace{-1cm} $\bullet$ some (e.g $\lambda = 0.1$) \\ \hspace{-1cm} led to marginal increases 
\item \hspace{-1cm} $\bullet$ others (e.g $\lambda = 5.5$) \\ \hspace{-1cm} separated slightly poorer 
\item \hspace{-1cm} $\bullet$ the differences were however \\ \hspace{-1cm} very minimal \& non-significant
\end{enumerate}
\end{columns}
\end{frame}

\begin{frame}{Transformations effects on treatments separation- Epilogue}
\begin{flushleft}
$\diamond$ In both methods, minimal \& insignificant differences were observed: Transformations failed to improve treatments separation 
 \\ \vspace{6mm} 
$\diamond$ Why? \\ \vspace{1mm}
\end{flushleft}
\end{frame}

\begin{frame}
\underline{1. Transformation effect on features distributions} \\
\centering
\includegraphics[height=8.5cm,width=9cm]{data1_distr_setdiff_pre_post_transformation_9to12.PDF}
\end{frame}

\begin{frame}
\underline{2. Differentiating ability of features selected (before transformation)} \\
\begin{columns}
\column{0.5\textwidth}
  \centering
\includegraphics[scale=0.45]{EDA_minimal_diffTRT.PDF}
\column{0.5\textwidth}
  \centering
  \includegraphics[scale=0.45]{EDA_maximal_diffTRT.PDF}
\end{columns}
\end{frame}


\begin{frame}{Effect of transformation on treatments active-calling}
\begin{columns}
\column{0.7\textwidth}
\flushleft
\includegraphics[scale=1.25]{optsfeat_active.PDF}
\column{0.3\textwidth}
\begin{enumerate}[ ]
\item \hspace{-1.5cm} $\bullet$ lower \% of active-calling
\item \hspace{-1.5cm} $\bullet$ intriguing relationship btwn  \\ \hspace{-1.5cm} number of features selected \\ \hspace{-1.5cm} \& prop. of active-calling
\item \hspace{-1.5cm} $\bullet$ similar relationship in 2$^{nd}$ \\ \hspace{-1.5cm} cell line
\end{enumerate}
\end{columns}
\end{frame}

\begin{frame}{Transform at cell- or well-level? [ T$^2$ \& AUC approaches]}
\begin{columns}
\column{0.5\textwidth}
% \underline{Hotelling's T$^2$} \\ \vspace{0.5mm}
\flushleft
\includegraphics[scale=0.45]{LOG_matched_aggregated_hotellings_allactiveTRTS_lambda=10point5.PDF}
\column{0.5\textwidth}
% \underline{AUC} \\ \vspace{1mm}
\flushleft
\includegraphics[scale=0.45]{aggregated_auc10point5.PDF}
\end{columns}
$\thicksim$ For $\lambda = 10.5$ \\
$\thicksim$ Minimal non-significant improvement in treatment separation when transformed at cell-level $>$ well level\\
$\thicksim$ High AUC values pre-transformation \\
$\thicksim$ No clear preference for cell- or well-level transformation \\
% $\thicksim$ No clear preference for cell- or well-level transformation \\
\vspace{1mm}
\end{frame}

\begin{frame}{Discussion}
From our study, we observed that: \\
$\thicksim$ Transformations did not improve treatments separation beyond what was seen pre-transformation \\
$\thicksim$ Transformations led to lower(higher) proportion of active-calling in 1$^{st}$(2$^{nd}$) data \\
$\thicksim$ Inverse relationship between proportion of active-calling and number of features selected was evident\\
$\thicksim$ There was no preference in transforming data at cell- or well-level 
\end{frame}
\input{LWafula_Thesis_Report_August2017_Presentation-concordance}
\end{document}



















